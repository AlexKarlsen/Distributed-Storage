\section{Erasure Coded Storage}
In the following two schemes is investigated; Case A where the lead node performs the encoding and Case B where the lead node delegates to another node to perform the encoding.\\
The time to encode and store, can be defined by the number of tolerated failures $k$, the data rate $R$, the size of the file in bits $s$, computation time $T_{process}$, and the time to encode $R_{enc}$ and decode $R_{dec}$.

\subsubsection*{Time to generate redundancy}
\textit{Case A:} 
\begin{align}
    T_{gen} = ...
\end{align}
\textit{Case B:} 
\begin{align}
    T_{gen} = ...
\end{align}

\subsubsection*{Time for lead node to finish}
\textit{Case A:} 
\begin{align}
    T_{lead} = R \cdot s + R_{enc} \cdot s + R \cdot ...
\end{align}
\textit{Case B:} 
\begin{align}
    T_{lead} = 2R \cdot s
\end{align}

\subsubsection*{Access time}
\begin{align}
    T_{access} = R \cdot ... + R_{dec} + R \cdot s
\end{align}

\subsubsection*{Measurements}
...