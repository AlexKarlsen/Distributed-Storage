\section{Replication}
In the following two schemes is investigated; Case A where the lead node distributes the file directly to all nodes and Case B where the lead node delegates to other nodes to generate the next replicas. \\
The time to generate all replicas, can be defined by the number of replicas $k$, the data rate $R$, the size of the file in bits $s$ and computation time $T_{process}$. 

\subsubsection*{Time to generate all replicas}
\textit{Case A:} 
\begin{align}
    T_{gen} = (k+1)\cdot R\cdot s + T_{process}
\end{align}
\textit{Case B:} 
\begin{align}
    T_{gen} = (k+1)\cdot R\cdot s + T_{process}
\end{align}

The time to to generate all the replicas for both schemes are more or less the same, since the amount of data to be transferred is the same. 

\subsubsection*{Time for lead node to finish}
\textit{Case A:}
\begin{align}
    T_{lead} = (k+1)\cdot R\cdot s + T_{process}
\end{align}
\textit{Case B:}
\begin{align}
    T_{lead} = 2R \cdot s + T_{process}
\end{align}
The lead node is freed earlier in the latter case. In this case it only has to send to one other node, instead of $k$.

\subsubsection*{Access time}
\begin{align}
    T_{access} = 2R \cdot s + T_{process}
\end{align}

\subsubsection*{Measurements}

\begin{table}[H]
	\centering
	\begin{tabular}{|l|l|l|l|}
		\hline
		\cellcolor{lightgray}\textbf{Size} & \cellcolor{lightgray}\textbf{Time Case A [s]} & \cellcolor{lightgray}\textbf{Time Case B [s] (k=2)} & \cellcolor{lightgray}\textbf{Time Case B [s] (k=3)} \\ \hline
		10KB & ...  & ... & ... \\ \hline
	\end{tabular}
	\caption{Measurements}
	\label{tab:e2meas}
\end{table}
