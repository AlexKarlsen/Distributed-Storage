\section{Introduction}
This paper is a mini-project in the Distributed Storage course at Aarhus University. The objective is to investigate different strategies for reliable file storage systems. The first strategy is file replication on multiple nodes in a cluster and the second strategy is erasure coded block storage using Random Linear Network Codes (RLNC). Each strategy is examined with two different schemes. The schemes differs how distribution of the data is handled. The mini-project includes implementation of the storage systems on a Raspberry Pi (RPi) cluster. The cluster consists of 4 RPi 2s interconnected with a network switch. Node 1 is the lead node, which exposes an REST API endpoints for upload and download using the different strategies to end-users. Node 2, 3 and 4 exposes REST APIs to the lead node with endpoints for the different strategies and schemes. The solution is written in \texttt{Python} using \texttt{Flask} \cite{flask}. \texttt{Flask} is a micro framework to develop REST APIs. The implementation of the two strategies and schemes will be measured using time metrics for upload and download of different file sizes. For erasure coding the overhead introduced by encoding and decoding using \texttt{kodo-python} \cite{kodo} are also measured. Kodo-python is an erasure coding library supporting RLNC. The two strategies and schemes will be evaluated by a comparison.

The rest of the paper is structured as follows; Section \ref{sec:repl} describes the replication strategy and the results obtained. Section \ref{sec:ec} describe the erasure coding strategy and the results obtained. Section \ref{sec:comp} compares the strategies. Section \ref{sec:dis} discusses different implementation approaches and assumptions made. Lastly section \ref{sec:conc} concludes upon the mini-project outcomes and learnings. 