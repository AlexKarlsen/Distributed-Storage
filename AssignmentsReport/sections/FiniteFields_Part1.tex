\section{Finite Fields - Part 1}

\subsection{Exercise 1: The Binary Extension Field}
\textbf{a) Fill in the missing values in the table below.}
\begin{table}[htbp]
    \begin{tabularx}{\textwidth}{| Y | Y | Y |}
        \hline
        \cellcolor{gray}\textbf{Polynomial}  & \cellcolor{gray}\textbf{Binary} & \cellcolor{gray}\textbf{Decimal} \\\hline
        \cellcolor{lightgray}$x^7+x^6+x^4+x+1$ & 11010011 & 211 \\\hline
        $x^7+x^6+x^3+1$ & \cellcolor{lightgray}11001001 & 201 \\\hline
        $x^7+x^2+1$ & 10000101 & \cellcolor{lightgray}133 \\\hline
        \cellcolor{lightgray}$x^4+x^2+x$ & 00010110 & 22 \\\hline
        $x^4+x^3+1$ & \cellcolor{lightgray}00011001 & 25 \\\hline
        $x^3+x$ & 00001010 & \cellcolor{lightgray}10 \\\hline
    \end{tabularx}
    \caption{$GF(2^m)$}
    \label{tab:ex1a}
\end{table}

\noindent\textbf{b) In the table, what is the required value for \textit{m} in order to represent the field elements}\\
\underline{Answer:} At least 8\\
\underline{Reason:} $GF(2^m)$ is a binary extension fields, and all field elements may be represented as binary polynomials of degree at most $m-1$. Since the highest polynomial degree is $x^7$, $m=7+1=8$

\subsection{Exercise 2: Polynomial Addition and Subtraction}
\textbf{a)}
\begin{align}
    f(x)  &= (x^5 + x) + (x^3 + x^2) \\
    &= x^5 + x^3 + x^2 + x
\end{align}
\indent\indent\indent or
\begin{align}
    &100010 \\
    \text{XOR } &001100 \\[-2ex] 
    \cline{1-2}
    &101110
\end{align}

\noindent\textbf{b)}
\begin{align}
    f(x)  &= (x^7 + x^3) + (x^7 + x + 1) \\
    &= (1 \oplus 1)x^7 + x^3 + x + 1\\
    &= x^3 + x + 1
\end{align}

\noindent\textbf{c)}
\begin{align}
    f(x)  &= (x^3 + x^2 + x + 1) + (x + 1) \\
    &= x^3 + x^2 + (1 \oplus 1)x + (1 \oplus 1)1\\
    &= x^3 + x^2
\end{align}

\noindent\textbf{d)}
\begin{align}
    f(x)  &= (x^7 + x^6 + x^5 + x^4 + x^3 + x^2 + 1) + (x^4 + x^2 + 1) \\
    &= x^7 + x^6 + x^5 + (1 \oplus 1)x^4 + x^3 + (1 \oplus 1)x^2 + (1 \oplus 1)1\\
    &= x^7 + x^6 + x^5 + x^3
\end{align}

\noindent In the binary extension field subtraction and addition are identical (based on the XOR).

\noindent\textbf{e)}
\begin{align}
    f(x)  &= (x^5 + x) - (x^3 + x^2) \\
    &= (x^5 + x) + (x^3 + x^2) \\
    &= x^5 + x^3 + x^2 + x
\end{align}

\noindent\textbf{f)}
\begin{align}
    f(x)  &= (x^7 + x^3) - (x^7 + x + 1) \\
    &= (x^7 + x^3) + (x^7 + x + 1) \\
    &= (1 \oplus 1)x^7 + x^3 + x + 1\\
    &= x^3 + x + 1
\end{align}

\noindent\textbf{g)}
\begin{align}
    f(x)  &= (x^3 + x^2 + x + 1) - (x + 1) \\
    &= (x^3 + x^2 + x + 1) + (x + 1) \\
    &= x^3 + x^2 + (1 \oplus 1)x + (1 \oplus 1)1\\
    &= x^3 + x^2
\end{align}

\noindent\textbf{h)}
\begin{align}
    f(x)  &= (x^7 + x^6 + x^5 + x^4 + x^3 + x^2 + 1) - (x^4 + x^2 + 1) \\
    &= (x^7 + x^6 + x^5 + x^4 + x^3 + x^2 + 1) + (x^4 + x^2 + 1) \\
    &= x^7 + x^6 + x^5 + (1 \oplus 1)x^4 + x^3 + (1 \oplus 1)x^2 + (1 \oplus 1)1\\
    &= x^7 + x^6 + x^5 + x^3
\end{align}

\pagebreak