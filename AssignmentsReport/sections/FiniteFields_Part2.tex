\section{Finite Fields - Part 2}

\subsection{Exercise 1}
For $GF(2^3)$ with primitive polynomial $x^3+x^2+1$\\

\noindent\textbf{a) Addition table}\\
...
\begin{table}[H]
    \begin{tabularx}{\textwidth}{| Y | Y | Y | Y | Y | Y | Y | Y | Y |}
        \hline
        \cellcolor{lightgray} & \cellcolor{lightgray}0 & \cellcolor{lightgray}1 & \cellcolor{lightgray}2 & \cellcolor{lightgray}3 & \cellcolor{lightgray}4 & \cellcolor{lightgray}5 & \cellcolor{lightgray}6 & \cellcolor{lightgray}7 \\\hline
        \cellcolor{lightgray}0 & 0 & 1 & 2 & 3 & 4 & 5 & 6 & 7 \\\hline
        \cellcolor{lightgray}1 & 1 & 0 & 3 & 2 & 5 & 4 & 7 & 6 \\\hline
        \cellcolor{lightgray}2 & 2 & 3 & 0 & 1 & 6 & 7 & 4 & 5 \\\hline
        \cellcolor{lightgray}3 & 3 & 2 & 1 & 0 & 7 & 6 & 5 & 4 \\\hline
        \cellcolor{lightgray}4 & 4 & 5 & 6 & 7 & 0 & 1 & 2 & 3 \\\hline
        \cellcolor{lightgray}5 & 5 & 4 & 7 & 6 & 1 & 0 & 3 & 2 \\\hline
        \cellcolor{lightgray}6 & 6 & 7 & 4 & 5 & 2 & 3 & 0 & 1 \\\hline
        \cellcolor{lightgray}7 & 7 & 6 & 5 & 4 & 3 & 2 & 1 & 0 \\\hline
    \end{tabularx}
    \caption{Addition table}
    \label{tab:ff21a}
\end{table}
\noindent\textbf{b) Multiplication table}\\
...
\begin{table}[H]
    \begin{tabularx}{\textwidth}{| Y | Y | Y | Y | Y | Y | Y | Y | Y |}
        \hline
        \cellcolor{lightgray} & \cellcolor{lightgray}0 & \cellcolor{lightgray}1 & \cellcolor{lightgray}2 & \cellcolor{lightgray}3 & \cellcolor{lightgray}4 & \cellcolor{lightgray}5 & \cellcolor{lightgray}6 & \cellcolor{lightgray}7 \\\hline
        \cellcolor{lightgray}0 & 0 & 0 & 0 & 0 & 0 & 0 & 0 & 0 \\\hline
        \cellcolor{lightgray}1 & 0 & 1 & 2 & 3 & 4 & 5 & 6 & 7 \\\hline
        \cellcolor{lightgray}2 & 0 & 2 & 4 & 6 & 5 & 7 & 1 & 3 \\\hline
        \cellcolor{lightgray}3 & 0 & 3 & 6 & 5 & 1 & 2 & 7 & 4 \\\hline
        \cellcolor{lightgray}4 & 0 & 4 & 5 & 1 & 7 & 3 & 2 & 6 \\\hline
        \cellcolor{lightgray}5 & 0 & 5 & 7 & 2 & 3 & 6 & 4 & 1 \\\hline
        \cellcolor{lightgray}6 & 0 & 6 & 1 & 7 & 2 & 4 & 3 & 5 \\\hline
        \cellcolor{lightgray}7 & 0 & 7 & 3 & 4 & 6 & 1 & 5 & 2 \\\hline
    \end{tabularx}
    \caption{Multiplication table}
    \label{tab:ff21b}
\end{table}

\subsection{Exercise 2}
\begin{align}
    G = 
    \begin{bmatrix}
        1 & 1 & 1 & 1 \\
        1 & 2 & 3 & 4 \\
        1 & 4 & 5 & 7 \\
        1 & 5 & 2 & 6 \\
        1 & 7 & 6 & 2 \\
        1 & 3 & 7 & 5
    \end{bmatrix}
\end{align}

\noindent\textbf{a) Select rows 1, 2, 5 and 6 and perform Gaussian Elimination on the resulting matrix.}\\
\begin{align}
    G = 
    \begin{bmatrix}
        1 & 1 & 1 & 1 \\
        1 & 2 & 3 & 4 \\
        1 & 7 & 6 & 2 \\
        1 & 3 & 7 & 5 \\
        1 & 4 & 5 & 7 \\
        1 & 5 & 2 & 6         
    \end{bmatrix}
\end{align}

\noindent\textbf{b) Calculate the transpose of G, i.e., $G^T$}\\
\begin{align}
    G_{sys} = G^T = 
    \begin{bmatrix}
        1 & 1 & 1 & 1 & 1 & 1 \\
        1 & 2 & 7 & 3 & 4 & 5 \\
        1 & 3 & 6 & 7 & 5 & 2 \\
        1 & 4 & 2 & 5 & 7 & 6 
    \end{bmatrix}
\end{align}

\noindent\textbf{c) Calculate Gaussian elimination on $G^T$ to obtain $G_{sys} = [I : R]$}\\
Subtracting row 1 from all other rows, to get 0's: \\
\begin{align}
    G_{sys} = 
    \begin{bmatrix}
        1 & 1 & 1 & 1 & 1 & 1 \\
        0 & 3 & 6 & 2 & 5 & 4 \\
        0 & 2 & 7 & 6 & 4 & 3 \\
        0 & 5 & 3 & 4 & 6 & 7 
    \end{bmatrix}
\end{align}
Multiplying: \\row 2 by 4 \\row 3 by 6 \\row 4 by 7, to get 1's in column 2, as seen in \ref{tab:ff21b}: \\
\begin{align}
    G_{sys} = 
    \begin{bmatrix}
        1 & 1 & 1 & 1 & 1 & 1 \\
        0 & 1 & 2 & 5 & 3 & 7 \\
        0 & 1 & 5 & 3 & 2 & 7 \\
        0 & 1 & 4 & 6 & 5 & 2 
    \end{bmatrix}
\end{align}
Subtracting row 2 from 3 and 4 to get 0's \\
\begin{align}
    G_{sys} = 
    \begin{bmatrix}
        1 & 1 & 1 & 1 & 1 & 1 \\
        0 & 1 & 2 & 5 & 3 & 7 \\
        0 & 0 & 7 & 6 & 1 & 0 \\
        0 & 0 & 6 & 3 & 6 & 5 
    \end{bmatrix}
\end{align}
Multiplying: \\row 3 by 5 \\row 4 by 2, to get 1's in column 3, as seen in \ref{tab:ff21b}: \\
\begin{align}
    G_{sys} = 
    \begin{bmatrix}
        1 & 1 & 1 & 1 & 1 & 1 \\
        0 & 1 & 2 & 5 & 3 & 7 \\
        0 & 0 & 1 & 4 & 5 & 0 \\
        0 & 0 & 1 & 6 & 1 & 7 
    \end{bmatrix}
\end{align}
Subtracting row 3 from 4 to get a 0 \\
\begin{align}
    G_{sys} = 
    \begin{bmatrix}
        1 & 1 & 1 & 1 & 1 & 1 \\
        0 & 1 & 2 & 5 & 3 & 7 \\
        0 & 0 & 1 & 4 & 5 & 0 \\
        0 & 0 & 0 & 2 & 4 & 7 
    \end{bmatrix}
\end{align}
Multiplying row 4 by 6 , to get a 1 in column 4, as seen in \ref{tab:ff21b}: \\
\begin{align}
    G_{sys} = 
    \begin{bmatrix}
        1 & 1 & 1 & 1 & 1 & 1 \\
        0 & 1 & 2 & 5 & 3 & 7 \\
        0 & 0 & 1 & 4 & 5 & 0 \\
        0 & 0 & 0 & 1 & 2 & 5 
    \end{bmatrix}
\end{align} 
Subtracting \\ row 4 times 4 $[$ 0 0 0 4 5 3 $]$ from row 3 \\row 4 times 5 $[$ 0 0 0 5 7 6 $]$ from row 2 \\ row 4 from row 1:\\
\begin{align}
    G_{sys} = 
    \begin{bmatrix}
        1 & 1 & 1 & 0 & 3 & 4 \\
        0 & 1 & 2 & 0 & 4 & 1 \\
        0 & 0 & 1 & 0 & 0 & 3 \\
        0 & 0 & 0 & 1 & 2 & 5 
    \end{bmatrix}
\end{align} 
Subtracting \\row 3 times 2 $[$ 0 0 2 0 0 6 $]$ from row 2 \\ row 3 from row 1:\\
\begin{align}
    G_{sys} = 
    \begin{bmatrix}
        1 & 1 & 0 & 0 & 3 & 7 \\
        0 & 1 & 0 & 0 & 4 & 7 \\
        0 & 0 & 1 & 0 & 0 & 3 \\
        0 & 0 & 0 & 1 & 2 & 5 
    \end{bmatrix}
\end{align} 
Subtracting row 2 from row 1:\\
\begin{align}
    G_{sys} = 
    \begin{bmatrix}
        1 & 0 & 0 & 0 & 7 & 0 \\
        0 & 1 & 0 & 0 & 4 & 7 \\
        0 & 0 & 1 & 0 & 0 & 3 \\
        0 & 0 & 0 & 1 & 2 & 5 
    \end{bmatrix}
\end{align} 

\noindent\textbf{d) Calculate the transpose of $G_{sys}$ to obtain a systematic RS generator matrix using the Vandermonde matrix.}\\
\begin{align}
    G = G_{sys}^T = 
    \begin{bmatrix}
        1 & 0 & 0 & 0 \\
        0 & 1 & 0 & 0 \\
        0 & 0 & 1 & 0 \\
        0 & 0 & 0 & 1 \\
        7 & 4 & 0 & 2 \\
        0 & 7 & 3 & 5
    \end{bmatrix}
\end{align}

\noindent\textbf{e) As in a previous example, select rows 1, 2, 5 and 6 of $G_{sys}$ and perform Gaussian Elimination on the resulting matrix.}\\
...

\pagebreak